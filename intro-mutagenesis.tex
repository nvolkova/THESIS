\begin{document}
\pagestyle{empty}
\section{Mutagenesis}

%%%%%%%%%%%%%%%%%%%%%%%%%%%%%%%%%%%%%%%%%%%%%%%%%%%%%%%%%%%%%

\subsection{History of mutagenesis research}

\todo[inline]{collect some reviews on mutagenesis in different model organisms}

%%%%%%%%%%%%%%%%%%%%%%%%%%%%%%%%%%%%%%%%%%%%%%%%%%%%%%%%%%%%%

\subsection{DNA repair and DNA damage}

\subsubsection{Types of DNA damage}

\subsubsection*{Environmental and intrinsic damage sources}

\subsubsection{DNA repair pathways and related diseases}

\subsubsection*{Direct repair}

\subsubsection*{Mismatch repair}

Fidelity of replication is maintained via multiple mechanisms: replicative polymerases 
Pol $\epsilon$ and Pol $\delta$ possess high selectivity against mismatches as well as a 
3'-exonuclease activity, and also backed up by a postreplicative repair pathway - DNA mismatch repair.
It is initiated by the recognition of replication errors left by the replicative polymerase, which 
has an error rate of about $10^{-5}$ to $10^{-4}$ (\cite{kunkel2000dna}) depending on the nucleotide,
local chromosomal properties and DNA polymerase.

The recognition of mismatches is performed by MutS protein complexes first found in bacteria *REF*.
In eukatyotes (???), there are two complexes: MutS$\alpha$, which consists of MSH2 and MSH6 proteins, 
and MutS$\beta$ cosisting of MSH2 and MSH3. Due to different mismatch binding sites, they have 
different substrate specificity: MutS-$\alpha$ preferentially detects base-base mismatches and short 1-2bp indels,
and MutS-$\beta$ handles larger indels up to 15 bp (\cite{Drummond1995-gc}, \cite{Habraken1996-vc}, 
\cite{Genschel1998-by}). MutS proteins form a clamp sliding along the genome, which is activated upon an
encounter with a mismatch and loads the other repair complex MutL to license excision.

%%%%%%% to be rewritten later %%%%%%%%%%%

Binding of MutS proteins to the DNA lesion facilitates subsequent recruitment of 
the MutL complex. MutL enhances mismatch recognition and 
promotes a conformational change in MutS through ATP hydrolysis to allow for the sliding of the MutL/MutS 
complex away from mismatched DNA (Allen, Gradia). DNA repair is initiated in most systems by a single-stranded 
nick generated by MutL (MutH in \textit{E. coli}) on the nascent DNA strand at some distance to the lesion 
(Kadyrov, Kadyrov2). Exonucleolytic activities in part conferred by Exo1 contribute to the removal of the DNA 
stretch containing the mismatch followed by gap filling via lagging strand DNA synthesis (Szankasi, Longley, 
Tishkoff, Genschel, Genschel and Modrich 2003, Constantin et al. 2005, Goellner et al. 2015). The most 
prominent MutL activity in human cells is provided by the MutL-$\alpha$ heterodimer, MLH1/PMS2 (Prolla et al. 1998; 
Cannavo et al. 2005). However, human MLH1 is also found in heterodimers with PMS1 and MLH3, called 
MutL-$\beta$ and MutL-$\gamma$. Of these two only MutL-$\gamma$ is thought to have a minor role in MMR 
(Cannavo et al. 2005). In \textit{C. elegans}, MLH-1 and PMS-2 are the only MutL homologs encoded in the genome.


A large number of studies have analyzed mutations arising in DNA mismatch repair deficient cells at specific 
genomic loci or in reporter constructs. Analysis of microsatellite loci in \textit{mlh1} deficient 
colorectal cancer cell lines suggested rates of repeat expansions or contractions between 
$8.4 \cdot 10^{-3}$ to $3.8 \cdot 10^{-2}$ per locus and generation 
(Bhattacharyya et al. 1994; Hanford et al. 1998). Estimates using \textit{S. cerevisiae} 
revealed a 100- to 700-fold increase in DNA repeat tract instability in \textit{pms2}, 
\textit{mlh1} and \textit{msh2} mutants (Strand et al. 1993) and a ~5-fold increase in 
base substitution rates (Yang et al. 1999). \textit{C. elegans} assays using reporter 
systems or selected, PCR-amplified regions revealed a more than 30-fold increased 
frequency of single base substitutions in msh-6, a 500-fold increase in mutations 
in A/T homopolymer runs and a 100-fold increase in mutations in dinucleotide repeats 
(Degtyareva et al. 2002; Tijsterman et al. 2002; Denver et al. 2005), akin to the 
frequencies observed in yeast and mammalian cells (Strand et al. 1993; Hanford et al. 1998). 
Recently, whole genome sequencing approaches using diploid S. cerevisiae started to provide 
a genome-wide view of MMR deficiency. \textit{S. cerevisiae} lines carrying an \textit{msh2} 
deletion alone or in conjunction with point mutations in one of the three replicative 
polymerases, Pol $\alpha$/primase, Pol-$\delta$, and Pol-$\epsilon$, were propagated over 
multiple generations to determine the individual contribution of replicative polymerases 
and MMR to replication fidelity (Lang et al. 2013; Lujan et al. 2014; Lujan et al. 2015). 
These analyses observed an average base substitution rate of 1.6 x 10-8 per base pair per 
generation in msh2 mutants and an increased rate in mutants in which \textit{msh2} and one 
of the replicative polymerases was mutated (Lujan et al. 2014; Lujan et al. 2015). 
A synergistic increase in mutagenesis was also recently observed in childhood tumors in 
which MMR deficiency and mutations in replicative polymerase $\epsilon$ and $\delta$ 
needed for leading and lagging strand DNA synthesis occurred \cite{Shlien}. 

\subsubsection*{Nucleotide excision repair}

\subsubsection*{Base excision repair}

\subsubsection*{Crosslink repair}

\subsubsection*{Single strand break repair}

\subsubsection*{Double strand break repair}

\subsubsection*{Translesion synthesis}

\subsubsection*{DNA damage response regulation}

%%%%%%%%%%%%%%%%%%%%%%%%%%%%%%%%%%%%%%%%%%%%%%%%%%%%%%%%%%%%%

\subsection{Methods for detection of DNA damage and DNA repair}

\end{document}