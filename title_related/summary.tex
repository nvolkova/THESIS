

\begin{document}
\pagestyle{empty}

\chapter*{Summary}

Maintenance of genome integrity via replication fidelity is essential to survival of any living organism. 
Evolution has supplied the cells with a range of repair mechanisms to defend the genetic information from 
a multitude of endogenous and exogenous mutagenic factors. Investigating the unique mutational patterns 
generated by these factors, i.e. the mutational signatures, may be of great use for understanding the 
origins of each individual tumour. However, the signatures identified so far mostly represent complex 
conglomerates of the action of different mutational processes. For many signatures, the link with the 
underlying mutational processes is still unclear.

Every mutational signature is a result of interaction between DNA damage and DNA repair pathways available 
in the cell. Using the data from \textit{C. elegans} mutagenesis screen with 90 DNA repair deficient genetic 
backgrounds and 13 genotoxins, we explored the diversity of damage-repair interactions. Out of over 200 
combinations of knockouts and genotoxins, over 35\% demonstrated a measurable effect on the mutation 
rate compared to expected values, and about 10\% also presented a new mutational spectrum. 

The controlled nature of the experiments also helps to better understand the functionality of DNA repair 
pathways. By introducing additional damage, we can assess the variability of repaired and unrepaired 
profiles. In the case of double \textit{pole-4; pms-2} knockout of mismatch repair (MMR) component and 
a subunit polymerase epsilon where we observed a 3-fold increase in the number of mutations compared to 
MMR mutants indicating that MMR system corrects the errors left by leading strand polymerase epsilon.

The patterns extracted from \textit{C. elegans} can be successfully translated into human cancer investigation, 
providing insights into mechanisms of signature genesis. Experimental signatures of mismatch repair 
deficiency derived from \textit{C. elegans} matched a mismatch repair associated signature found in 
gastric cancers. An insight from \textit{C. elegans} mutational profile suggesting the use of 1-bp 
insertions and deletions as a part of MMR signature also helped to refine the human MMR deficiency 
profile and separate it from other mutational processes.

This thesis also suggests damage-repair interactions as a phenomenon driving variability of mutational 
spectra across human cancers, although cases of hypermutation are surprisingly rare despite signs of 
positive selection in a number of DNA repair genes. Nevertheless, cancer risk may be substantially 
elevated even by small increases in mutagenicity according to evolutionary multi-hit theory. Overall, 
our data underscore how mutagenesis is a joint product of DNA damage and DNA repair, implying that 
mutational signatures may be more variable than currently anticipated.

\end{document}