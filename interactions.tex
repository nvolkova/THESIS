\begin{document}
\pagestyle{empty}

\section{Interactions between DNA damage and repair}

Mechanistic motivation: spectrum = damage - repair



%%%%%%%%%%%%%%%%%%%%%%%%%%%%%%%%%%%%%%%%%%


\subsection{Methods}

\subsubsection{Models for effect extraction: MCMC and optimization}

\subsubsection{Comparison of efficiency and accuracy}


%%%%%%%%%%%%%%%%%%%%%%%%%%%%%%%%%%%%%%%%%%

\subsection{Alteration of mutagen profiles in \textit{C. elegans} experiments}

Summaries of interactions: common and quite strong

\subsubsection{Non-additive effects}

polk-1 and mutagens

agt-1 and EMS/MMS

\subsubsection{Translesion synthesis and mutagens}

TLS polymerases

\subsubsection{Changes in effect size}

NER mutants and size effect foldchage

%%%%%%%%%%%%%%%%%%%%%%%%%%%%%%%%%%%%%%%%%%

\subsection{Evidence of interactions from other sources}

Data from other model experiments

Metabolic activation of mutagens in mammals

Simulation of damage is far from real damage

Methods for damage recognition

%%%%%%%%%%%%%%%%%%%%%%%%%%%%%%%%%%%%%%%%%%


\subsection{Discussion}

What could be done: damage sequencing with Nanopore or other things


\end{document}