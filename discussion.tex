

\begin{document}
\pagestyle{empty}
\chapter{Discussion}


Matching mutational signatures to DNA repair deficiency has a tremendous potential to stratify cancer therapy tailored to DNA repair deficiency. This approach appears advantageous over genotyping marker genes, as mutational signatures provide a read-out for cellular repair deficiency associated with either genetic or epigenetic defects. Following on from our study we expect that analyzing DNA repair defective model organisms and human cell lines, alone or in conjunction with defined genotoxic agents, will contribute to a more precise definition of mutational signatures occurring in cancer genomes and to establishing the etiology of these signatures.



Significance for mutagenesis field: comprehensive catalogue of high-resolution profiles for a wide range of genetic knock-outs and mutagens

Significance for clinical research: revealing complicated relationships between mutagenic processes and DNA repair pathway status

Implications for diagnostics, treatment, resistance

Signifies limitations to signature analysis in cancers

Overall limitations: low signal in worm data, variance, high noise in human data, timing and clonality issues, technological issues

Future work


\end{document}