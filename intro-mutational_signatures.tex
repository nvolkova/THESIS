
\begin{document}
\pagestyle{empty}

%%%%%%%%%%%%%%%%%%%%%%%%%%%%%%%%%%%%%%%%%%%%%%%%%%%%%%%%%%%%%

\section{Mutational signatures}

%%%%%%%%%%%%%%%%%%%%%%%%%%%%%%%%%%%%%%%%%%%%%%%%%%%%%%%%%%%%%

\subsection{Somatic mutations in cancer}

\subsection*{Mutagen exposure, DNA repair deficiencies and cancer risk}

\subsection*{Driver and passenger mutations}



%%%%%%%%%%%%%%%%%%%%%%%%%%%%%%%%%%%%%%%%%%%%%%%%%%%%%%%%%%%%%


\subsection{Concept of mutational signatures}

\subsection*{Individual factor analysis in human cancer }

Signatures inferred from model organisms are essential for the understanding of mutagenesis mechanisms. 
Apart from the significance for basic research, it would be of great importance to  translate these 
effects into human data as it would allow to study the causes of the diseases caused by acquired 
changes in DNA, and first of all of cancer.

The main difficulty consists in constructing an adequate transformation of the signatures 
obtained using worm data in order to apply them to other organism. We are planning to 
normalize the signatures as much as possible using functional and chemical features 
available for \textit{C. elegans} and further validate them by the use of yeast and 
human iPS cell lines data from COMSIG.

Using these translated signatures, we plan to perform single mutagenic factor based 
analysis of the primary cancer samples from TCGA. There are two strategies we are 
going to apply in order to establish the links between mutational factor signatures 
and tumor DNA alteration spectra:

\begin{itemize}
\item Investigate the samples with known exposures and DNA repair deficiencies in 
order to estimate individual contributions of different factors into the overall 
mutational landscape of each cancer sample. 
\item Try to identify the patterns caused by individual factors in primary cancer 
samples and link them back to the cancer types the samples are coming from.
\end{itemize}

After proving applicability of the individual factor signatures to human data,
we will be able to start a deeper investigation of mutational signatures in 
cancers and their underlying mechanisms.


\subsection*{Cancer genome decomposition}

According to the current state of the Catalogue Of Somatic Mutations In Cancer (COSMIC) project
(\cite{Forbes2015-gf}), there are 30 base substitution signatures identified so far (several indel 
and genomic rearrangement signatures identified recently (e.g. \cite{Alexandrov2015-clock}, \cite{Nik-Zainal2012-eg}) are 
not included in the catalogue yet). It is important to notice that these signatures may be 
not unique: they were extracted from a large dataset of cancer genomes as one possible basis 
of the latent variable space, and their linear combination may represent more meaningful 
signatures (e.g., as was proven for signatures 1A, 1B and 5 in \cite{Alexandrov2015-clock}), expecially as 
soon as more data is added to the pool, increasing the statistical power of blind source 
separation methods such as NMF (\cite{Alexandrov2013-zu}).

In human cancer samples 30 mutational signatures (referred to as COSMIC signatures 
from here on) have been uncovered by mathematical modeling across a large number of cancer genomes representing more
than 30 tumor types (http://cancer.sanger.ac.uk/cosmic/signatures) (Alexandrov 2013a; 
Alexandrov 2013b). These signatures are largely define by the relative frequency of the six possible 
base substitutions (C>A, C>G, C>T, T>A, T>C, T>G), occurring in a sequence context defined by their
adjacent 5’ and 3’ base. Of these, COSMIC signatures 6, 15, 20, 21 and 26, have been associated 
with MMR deficiency with several MMR signatures being present in the same tumor sample.

Based on large cohort studies, several significant associations were identified for 
17 base substitution signatures with both genetic and environmental factors (\cite{Alexandrov2013-md},\cite{Alexandrov2015-clock},\cite{Roberts2014-nl}):

\begin{itemize}
\itemsep0em
\item Genetic factors:
  \begin{itemize}
  \itemsep0em
  \item Signatures 1 and 5 are strongly correlated with age and may therefore represent aging related mutational processes (\cite{Alexandrov2015-clock});
  \item Signatures 2 and 13 were associated with the activity of APOBEC family related pathways;
  \item Signature 3 - with defective homologous recombination;
  \item Signatures 6, 15, 20 and 26 - with DNA mismatch repair deficiency (\cite{Supek2015-pr});
  \item Signatures 9 and 10 are associated with polymerases failures;
  \end{itemize}
\item Mutagen exposures:
  \begin{itemize}
  \itemsep0em
  \item Signature 24 has a proposed link to aflatoxin exposure;
  \item Signatures 4 and 29 have been attributed to tobacco mutagenesis;
  \item Signature 7 is potentially associated with UV;
  \item Signature 11 - with the action of an alkylating agent temozolomide (\cite{Poon2014-review});
  \item Signature 22 was only found in the samples with aristolochic acid exposure (\cite{Poon2015-AA});
  \end{itemize}
\end{itemize}

In order to further explain the mutational background of cancer, we propose to perform the following steps:

\begin{enumerate}
\item Assess cancer signature purity by direct comparison of individual factor signatures with the respective cancer-type dependent signature combinations.
\item Search for the further associations between the signatures and individual genetic and mutagenic components from the catalogue, measure the contribution of interaction factors.
\item Potentially recombine the signatures into new linear combinations which would be explicitly linked to a particular mechanism.
\item Decompose the mutational spectra in different cancers based on individual factors identified in the first stage of this project, i.e. perform the NMF of a big cancer dataset using the effects catalogue as a basis, and compare the obtained contribution patterns to known causal relationships.
\end{enumerate}

Additionally, the information about interaction signatures will allow to recover the 
exposure history behind the mutational processes happened in the past, and will also 
help to predict the progress of an ongoing process and the quantitative results of a 
radio- or chemotoxin exposure of a particular tumor (\cite{Hollstein2017-ag}).

If we find a robust and stable basis combinations of exposures that will explain the 
causes and mutagenic contribution in different cancer types, they may serve as a clear 
and intuitive mean of cancer genome analysis. Prospectively we could also create a tool 
that would decompose an individual 
somatic mutational spectrum of a patient into corresponding signatures contributions and 
also identify the effects of particular factors.

Such representation may also allow to investigate the situations when the phenotype of a 
sample reports a genetic background which is not actually present (as samples with and 
without BRCA-1 mutations in breast cancers, \cite{NZ2}) by finding alternative combinations 
of factors leading to a similar observed pattern.

The investigation of these aspects is planned in 2018, followed by validation in additional 
primary samples datasets (TCGA, CGP, ICGC and other cancer data resources \cite{Tian}) and 
finalized with a wrap-up and thesis completion by April 2019.




%%%%%%%%%%%%%%%%%%%%%%%%%%%%%%%%%%%%%%%%%%%%%%%%%%%%%%%%%%%%%

\subsection{Methods for signature extraction}

\subsubsection*{Factor analysis}

\subsubsection*{Non-negative matrix factorisation}

\subsubsection*{Bayesian approaches and Dirichlet processes}



%%%%%%%%%%%%%%%%%%%%%%%%%%%%%%%%%%%%%%%%%%%%%%%%%%%%%%%%%%%%%

\subsection{Clinical applications of mutational signatures}




\end{document}