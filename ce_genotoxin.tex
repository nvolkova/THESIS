
\begin{document}
\pagestyle{empty}

\chapter{Experimental signatures of genotoxin exposure}

\section{Introduction}

\todo[inline]{transition from dna repair and mmr study into genotoxins in WT}

\subsection*{Contributions}

%%%%%%%%%%%%%%%%%%%%%%%%%%%%%%%%%%%%%%%%%%%%%%%%%%%%%%%%%%%%%%%%%%%%%%%%%%%%%%%%%%%%%%%%%%%%%%%%%%%

\section{Experimental signatures of genotoxic substances}

\subsection{Types of genotoxins}

Using this approach, we calculated the qualitative effect estimations for all of the factors. Effects for genotoxins used in the study are consistent with their chemical interactions with DNA known from literature (\cite{Helleday}, \cite{DNAdamagerepair}, \cite{Meier1}) and can be found in Figure N.

\subsection{Signatures of alkylating agents}

\subsection{Agents introducing bulky adducts}

\subsubsection{Aflatoxin}

\subsubsection{Aristolochic acid}

\subsection{Intercalating agents}

\subsubsection{Cisplatin}

\subsection{Irradiations}

\subsubsection{X-rays and $\gamma$ irradiation}

\subsubsection{UV}

%%%%%%%%%%%%%%%%%%%%%%%%%%%%%%%%%%%%%%%%%%%%%%%%%%%%%%%%%%%%%%%%%%%%%%%%%%%%%%%%%%%%%%%%%%%%%%%%%%%


\section{Relation to transcription and localisation}

%epigenetic features as well?

\subsection{Clustering of MMS mutations}

\subsection{Transcription strand bias for UV}

%%%%%%%%%%%%%%%%%%%%%%%%%%%%%%%%%%%%%%%%%%%%%%%%%%%%%%%%%%%%%%%%%%%%%%%%%%%%%%%%%%%%%%%%%%%%%%%%%%%

\section{Cross-species comparison}

Comparison between experimental signatures of 12 mutagens extracted from \textit{C.elegans} 
experiments to the catalogue of mutational signatures of environmental carcinogens from \cite{Kucab2019-ti}, 
and the catalogue of mutational signatures in cancer \cite{Alexandrov2018-ya} showed an average similarity of 
. \textit{C.elegans} signatures were adjusted to the human genome trinucleotide frequency.

Cosine distances between the humanized \textit{C.elegans} signatures and experimental (blue) 
and computational (red) signatures of same or related mutagens in human cells or cancers. 
"Experimental" signature for ionizing radiation was estimated as an averaged spectrum of 
12 radiotherapy-associated secondary malignancies from \cite{Behjati}. No counterparts 
from human data were found for HU, Mitomycin C and MMS. Signature of aflatoxin was similar 
to the computationally extracted signature 24 (similarity 0.8) but different from 
experimental one (0.62). UV-B exposure in worms showed a C>T mutation spectrum somewhat 
similar to that in cell line experiments and in cancer, albeit with an additional fraction
of T>C mutations. Signature of EMS in worms is very different from temozolomide signature 
in cell lines, but very similar (0.90) to cancer-derived computational signature SBS11 
associated with temozolomide. Interestingly, similar phenomena was observed upon exposure 
with alkylating agents in Salmonella typhimurium (Matsumura et al. 2018). Aristolochic 
acid signatures were consistent between both experimental systems and cancer. Signature 
of cisplatin was different from the one identified in cancer or human cell lines. 
Ionizing radiation and X-rays yielded profiles similar to those in human cancers. 
Exposures to mechlorethamine and DMS exhibited different mutational spectra in 
the two model systems.



Relevance to cancer

DNA repair pathways

Mutagens, cancers related to environmental exposure

%%%%%%%%%%%%%%%%%%%%%%%%%%%%%%%%%%%%%%%%%%%%%%%%%%%%%%%%%%%%%%%%%%%%%%%%%%%%%%%%%%%%%%%%%%%%%%%%%%%

\section{Discussion}

\end{document}