
\begin{document}
\pagestyle{empty}

\chapter{Introduction}

Genome stability maintenance plays the key role in cell’s ability to propagate. Spontaneous 
changes and mutagenic damage in the genome may cause wide range of diseases including cancer. 
These alterations can be an effect of combination of environmental factors damaging DNA and 
deficiencies in DNA repair and replication leading to characteristic mutational spectra.

Various mutational process attack the genome due to endogenous or exogenous factors, and 
many of them are believed to leave a unique pattern of mutations in the genome. Given the 
explosion in the amount of publicly available cancer sequencing data, it was tempting to 
aggregate it together and apply machine learning methods to extract the signatures of 
underlying mutagenic processes. Hence, mutational signatures (\cite{Alexandrov2013-md}, 
\cite{Alexandrov2015-clock}, \cite{Nik-Zainal2012-eg}) became a very useful tool of cancer 
investigation in the last years. However, most of the signatures identified so far represent 
complex conglomerates of the action of different mutational processes, deciphering which will 
be essential for understanding the causes of the disease and the most efficient treatment 
options. For some of the signatures, significant associations with various mutational 
processes were identified, although the precise causation is still unclear (\cite{Alexandrov2013-md},
\cite{Alexandrov2015-clock}, \cite{Helleday2014-kw}). For others, no links with the underlying 
mutational processes were found so far. 

The mutational profiles observed in sequencing studies represent the result of interaction 
between damaging and repairing factors, and the ability to disentangle the interactions 
between them which are shaping mutational history of a tumor gives more insight into the 
origins of a particular tumor and may suggest prospective targets for treatment 
(\cite{Alexandrov2013-zu}, \cite{Alexandrov2015-gastric}, \cite{Hollstein2017-ag}, \cite{Poon2014-review}). 
Dissecting the individual contributions of each of DNA repair deficiencies and external 
exposures represents a highly complex task due to  redundancy in DNA repair pathways, 
interaction effects, sequencing artifacts, and high inter- and intra-personal variance 
among patients. Hence the investigation of mutational patterns in a controlled environment 
of a model organism can give a better understanding of underlying biology and shed light 
on the precise contribution of the factors involved (\cite{Meier2014-aa}, \cite{Segovia2015-cr}).

DNA repair mechanisms represent the key component of genome stability maintenance. 
Repair ability deficiencies provide space for major alterations of the genome, both 
by the aggregation of spontaneous mutations and the lack of mutagen resistance. 
There are several major classes of DNA repair pathways: direct reversal, single-strand 
damage repair (such as base excision repair, nucleotide excision repair and mismatch 
repair systems), double-strand break repair (which includes homologous recombination, 
microhomology-mediated end joining and non-homologous end-joining), translesion synthesis, 
and also DNA damage signaling pathways and associated cell cycle control mechanisms 
(\cite{Alberts2007-qn}). Dissecting the individual contribution of each of them represents 
a highly complex task, since many DNA repair components have functions associated with 
different repair pathways, and in some cases these mechanisms may be interchangeable. 
Nevertheless it is possible to investigate the contributions of the most essential 
genetic components and their interplay with each other and with various cytotoxins 
to imitate the processes happening in cancer (\cite{Helleday2014-kw}, \cite{Wu2016-qp}).

The central conceptual parameter of cancer development are the driver genes that 
cause the most important events in individual cancer evolution (\cite{Stratton2009-kh}). 
The set of driver mutations is usually small, but most of them happen in the genes 
essential for cell growth or damage-caused apoptosis (such as tumor suppressor gene 
P53 (\cite{Rivlin2011-cz}) or oncogene KRAS (\cite{Wang2015-ki}) and lead to bursts of mutations directed 
by the effectiveness of particular DNA repair mechanisms. Knowing the individual and 
interaction effects of key elements of genome stability maintenance and environmental 
exposures will provide means to trace back the causes of the disease and predict the 
outcomes of chemo- or radiotherapy in every individual case (\cite{Poon2014-review}).

%%%%%%%%%%%%%%%%%%%%%%%%%%%%%%%%%%%%%%%%%%%%%%%%%

\begin{document}
\pagestyle{empty}
\section{Mutagenesis}

%%%%%%%%%%%%%%%%%%%%%%%%%%%%%%%%%%%%%%%%%%%%%%%%%%%%%%%%%%%%%

\subsection{History of mutagenesis research}

\todo[inline]{collect some reviews on mutagenesis in different model organisms}

%%%%%%%%%%%%%%%%%%%%%%%%%%%%%%%%%%%%%%%%%%%%%%%%%%%%%%%%%%%%%

\subsection{DNA repair and DNA damage}

\subsubsection{Types of DNA damage}

\subsubsection*{Environmental and intrinsic damage sources}

\subsubsection{DNA repair pathways and related diseases}

\subsubsection*{Direct repair}

\subsubsection*{Mismatch repair}

Fidelity of replication is maintained via multiple mechanisms: replicative polymerases 
Pol $\epsilon$ and Pol $\delta$ possess high selectivity against mismatches as well as a 
3'-exonuclease activity, and also backed up by a postreplicative repair pathway - DNA mismatch repair.
It is initiated by the recognition of replication errors left by the replicative polymerase, which 
has an error rate of about $10^{-5}$ to $10^{-4}$ (\cite{kunkel2000dna}) depending on the nucleotide,
local chromosomal properties and DNA polymerase.

The recognition of mismatches is performed by MutS protein complexes first found in bacteria *REF*.
In eukatyotes (???), there are two complexes: MutS$\alpha$, which consists of MSH2 and MSH6 proteins, 
and MutS$\beta$ cosisting of MSH2 and MSH3. Due to different mismatch binding sites, they have 
different substrate specificity: MutS-$\alpha$ preferentially detects base-base mismatches and short 1-2bp indels,
and MutS-$\beta$ handles larger indels up to 15 bp (\cite{Drummond1995-gc}, \cite{Habraken1996-vc}, 
\cite{Genschel1998-by}). MutS proteins form a clamp sliding along the genome, which is activated upon an
encounter with a mismatch and loads the other repair complex MutL to license excision.

%%%%%%% to be rewritten later %%%%%%%%%%%

Binding of MutS proteins to the DNA lesion facilitates subsequent recruitment of 
the MutL complex. MutL enhances mismatch recognition and 
promotes a conformational change in MutS through ATP hydrolysis to allow for the sliding of the MutL/MutS 
complex away from mismatched DNA (Allen, Gradia). DNA repair is initiated in most systems by a single-stranded 
nick generated by MutL (MutH in \textit{E. coli}) on the nascent DNA strand at some distance to the lesion 
(Kadyrov, Kadyrov2). Exonucleolytic activities in part conferred by Exo1 contribute to the removal of the DNA 
stretch containing the mismatch followed by gap filling via lagging strand DNA synthesis (Szankasi, Longley, 
Tishkoff, Genschel, Genschel and Modrich 2003, Constantin et al. 2005, Goellner et al. 2015). The most 
prominent MutL activity in human cells is provided by the MutL-$\alpha$ heterodimer, MLH1/PMS2 (Prolla et al. 1998; 
Cannavo et al. 2005). However, human MLH1 is also found in heterodimers with PMS1 and MLH3, called 
MutL-$\beta$ and MutL-$\gamma$. Of these two only MutL-$\gamma$ is thought to have a minor role in MMR 
(Cannavo et al. 2005). In \textit{C. elegans}, MLH-1 and PMS-2 are the only MutL homologs encoded in the genome.


A large number of studies have analyzed mutations arising in DNA mismatch repair deficient cells at specific 
genomic loci or in reporter constructs. Analysis of microsatellite loci in \textit{mlh1} deficient 
colorectal cancer cell lines suggested rates of repeat expansions or contractions between 
$8.4 \cdot 10^{-3}$ to $3.8 \cdot 10^{-2}$ per locus and generation 
(Bhattacharyya et al. 1994; Hanford et al. 1998). Estimates using \textit{S. cerevisiae} 
revealed a 100- to 700-fold increase in DNA repeat tract instability in \textit{pms2}, 
\textit{mlh1} and \textit{msh2} mutants (Strand et al. 1993) and a ~5-fold increase in 
base substitution rates (Yang et al. 1999). \textit{C. elegans} assays using reporter 
systems or selected, PCR-amplified regions revealed a more than 30-fold increased 
frequency of single base substitutions in msh-6, a 500-fold increase in mutations 
in A/T homopolymer runs and a 100-fold increase in mutations in dinucleotide repeats 
(Degtyareva et al. 2002; Tijsterman et al. 2002; Denver et al. 2005), akin to the 
frequencies observed in yeast and mammalian cells (Strand et al. 1993; Hanford et al. 1998). 
Recently, whole genome sequencing approaches using diploid S. cerevisiae started to provide 
a genome-wide view of MMR deficiency. \textit{S. cerevisiae} lines carrying an \textit{msh2} 
deletion alone or in conjunction with point mutations in one of the three replicative 
polymerases, Pol $\alpha$/primase, Pol-$\delta$, and Pol-$\epsilon$, were propagated over 
multiple generations to determine the individual contribution of replicative polymerases 
and MMR to replication fidelity (Lang et al. 2013; Lujan et al. 2014; Lujan et al. 2015). 
These analyses observed an average base substitution rate of 1.6 x 10-8 per base pair per 
generation in msh2 mutants and an increased rate in mutants in which \textit{msh2} and one 
of the replicative polymerases was mutated (Lujan et al. 2014; Lujan et al. 2015). 
A synergistic increase in mutagenesis was also recently observed in childhood tumors in 
which MMR deficiency and mutations in replicative polymerase $\epsilon$ and $\delta$ 
needed for leading and lagging strand DNA synthesis occurred \cite{Shlien}. 

\subsubsection*{Nucleotide excision repair}

\subsubsection*{Base excision repair}

\subsubsection*{Crosslink repair}

\subsubsection*{Single strand break repair}

\subsubsection*{Double strand break repair}

\subsubsection*{Translesion synthesis}

\subsubsection*{DNA damage response regulation}

%%%%%%%%%%%%%%%%%%%%%%%%%%%%%%%%%%%%%%%%%%%%%%%%%%%%%%%%%%%%%

\subsection{Methods for detection of DNA damage and DNA repair}

\end{document}

%%%%%%%%%%%%%%%%%%%%%%%%%%%%%%%%%%%%%%%%%%%%%%%%%


\begin{document}
\pagestyle{empty}

%%%%%%%%%%%%%%%%%%%%%%%%%%%%%%%%%%%%%%%%%%%%%%%%%%%%%%%%%%%%%

\section{Mutational signatures}

%%%%%%%%%%%%%%%%%%%%%%%%%%%%%%%%%%%%%%%%%%%%%%%%%%%%%%%%%%%%%

\subsection{Somatic mutations in cancer}

\subsection*{Mutagen exposure, DNA repair deficiencies and cancer risk}

\subsection*{Driver and passenger mutations}



%%%%%%%%%%%%%%%%%%%%%%%%%%%%%%%%%%%%%%%%%%%%%%%%%%%%%%%%%%%%%


\subsection{Concept of mutational signatures}

\subsection*{Individual factor analysis in human cancer }

Signatures inferred from model organisms are essential for the understanding of mutagenesis mechanisms. 
Apart from the significance for basic research, it would be of great importance to  translate these 
effects into human data as it would allow to study the causes of the diseases caused by acquired 
changes in DNA, and first of all of cancer.

The main difficulty consists in constructing an adequate transformation of the signatures 
obtained using worm data in order to apply them to other organism. We are planning to 
normalize the signatures as much as possible using functional and chemical features 
available for \textit{C. elegans} and further validate them by the use of yeast and 
human iPS cell lines data from COMSIG.

Using these translated signatures, we plan to perform single mutagenic factor based 
analysis of the primary cancer samples from TCGA. There are two strategies we are 
going to apply in order to establish the links between mutational factor signatures 
and tumor DNA alteration spectra:

\begin{itemize}
\item Investigate the samples with known exposures and DNA repair deficiencies in 
order to estimate individual contributions of different factors into the overall 
mutational landscape of each cancer sample. 
\item Try to identify the patterns caused by individual factors in primary cancer 
samples and link them back to the cancer types the samples are coming from.
\end{itemize}

After proving applicability of the individual factor signatures to human data,
we will be able to start a deeper investigation of mutational signatures in 
cancers and their underlying mechanisms.


\subsection*{Cancer genome decomposition}

According to the current state of the Catalogue Of Somatic Mutations In Cancer (COSMIC) project
(\cite{Forbes2015-gf}), there are 30 base substitution signatures identified so far (several indel 
and genomic rearrangement signatures identified recently (e.g. \cite{Alexandrov2015-clock}, \cite{Nik-Zainal2012-eg}) are 
not included in the catalogue yet). It is important to notice that these signatures may be 
not unique: they were extracted from a large dataset of cancer genomes as one possible basis 
of the latent variable space, and their linear combination may represent more meaningful 
signatures (e.g., as was proven for signatures 1A, 1B and 5 in \cite{Alexandrov2015-clock}), expecially as 
soon as more data is added to the pool, increasing the statistical power of blind source 
separation methods such as NMF (\cite{Alexandrov2013-zu}).

In human cancer samples 30 mutational signatures (referred to as COSMIC signatures 
from here on) have been uncovered by mathematical modeling across a large number of cancer genomes representing more
than 30 tumor types (http://cancer.sanger.ac.uk/cosmic/signatures) (Alexandrov 2013a; 
Alexandrov 2013b). These signatures are largely define by the relative frequency of the six possible 
base substitutions (C>A, C>G, C>T, T>A, T>C, T>G), occurring in a sequence context defined by their
adjacent 5’ and 3’ base. Of these, COSMIC signatures 6, 15, 20, 21 and 26, have been associated 
with MMR deficiency with several MMR signatures being present in the same tumor sample.

Based on large cohort studies, several significant associations were identified for 
17 base substitution signatures with both genetic and environmental factors (\cite{Alexandrov2013-md},\cite{Alexandrov2015-clock},\cite{Roberts2014-nl}):

\begin{itemize}
\itemsep0em
\item Genetic factors:
  \begin{itemize}
  \itemsep0em
  \item Signatures 1 and 5 are strongly correlated with age and may therefore represent aging related mutational processes (\cite{Alexandrov2015-clock});
  \item Signatures 2 and 13 were associated with the activity of APOBEC family related pathways;
  \item Signature 3 - with defective homologous recombination;
  \item Signatures 6, 15, 20 and 26 - with DNA mismatch repair deficiency (\cite{Supek2015-pr});
  \item Signatures 9 and 10 are associated with polymerases failures;
  \end{itemize}
\item Mutagen exposures:
  \begin{itemize}
  \itemsep0em
  \item Signature 24 has a proposed link to aflatoxin exposure;
  \item Signatures 4 and 29 have been attributed to tobacco mutagenesis;
  \item Signature 7 is potentially associated with UV;
  \item Signature 11 - with the action of an alkylating agent temozolomide (\cite{Poon2014-review});
  \item Signature 22 was only found in the samples with aristolochic acid exposure (\cite{Poon2015-AA});
  \end{itemize}
\end{itemize}

In order to further explain the mutational background of cancer, we propose to perform the following steps:

\begin{enumerate}
\item Assess cancer signature purity by direct comparison of individual factor signatures with the respective cancer-type dependent signature combinations.
\item Search for the further associations between the signatures and individual genetic and mutagenic components from the catalogue, measure the contribution of interaction factors.
\item Potentially recombine the signatures into new linear combinations which would be explicitly linked to a particular mechanism.
\item Decompose the mutational spectra in different cancers based on individual factors identified in the first stage of this project, i.e. perform the NMF of a big cancer dataset using the effects catalogue as a basis, and compare the obtained contribution patterns to known causal relationships.
\end{enumerate}

Additionally, the information about interaction signatures will allow to recover the 
exposure history behind the mutational processes happened in the past, and will also 
help to predict the progress of an ongoing process and the quantitative results of a 
radio- or chemotoxin exposure of a particular tumor (\cite{Hollstein2017-ag}).

If we find a robust and stable basis combinations of exposures that will explain the 
causes and mutagenic contribution in different cancer types, they may serve as a clear 
and intuitive mean of cancer genome analysis. Prospectively we could also create a tool 
that would decompose an individual 
somatic mutational spectrum of a patient into corresponding signatures contributions and 
also identify the effects of particular factors.

Such representation may also allow to investigate the situations when the phenotype of a 
sample reports a genetic background which is not actually present (as samples with and 
without BRCA-1 mutations in breast cancers, \cite{NZ2}) by finding alternative combinations 
of factors leading to a similar observed pattern.

The investigation of these aspects is planned in 2018, followed by validation in additional 
primary samples datasets (TCGA, CGP, ICGC and other cancer data resources \cite{Tian}) and 
finalized with a wrap-up and thesis completion by April 2019.




%%%%%%%%%%%%%%%%%%%%%%%%%%%%%%%%%%%%%%%%%%%%%%%%%%%%%%%%%%%%%

\subsection{Methods for signature extraction}

\subsubsection*{Factor analysis}

\subsubsection*{Non-negative matrix factorisation}

\subsubsection*{Bayesian approaches and Dirichlet processes}



%%%%%%%%%%%%%%%%%%%%%%%%%%%%%%%%%%%%%%%%%%%%%%%%%%%%%%%%%%%%%

\subsection{Clinical applications of mutational signatures}




\end{document}

%%%%%%%%%%%%%%%%%%%%%%%%%%%%%%%%%%%%%%%%%%%%%%%%%

\section{Aims and concepts of this thesis}

In this thesis, I will describe the experimental mutational signatures of various mutagenic 
agents and DNA repair deficiencies in \textit{C. elegans}, discuss the interplay between 
DNA damage and repair which may result in a dramatic change of mutational patterns in the 
genome, compare experimental results to human cancers and propose strategies for application 
of model organism-derived signatures to cancer signature investigation. The aim of 
this first chapter is to present an overview of processes involved in DNA damage acquisition 
and processing, give a summary of the methods previously developed for analysis of mutational 
signatures in cancer genomes, and describe the current state of DNA repair biology and cancer 
genomics fields and their understanding of mutational signatures.

%%This work was carried on in collaboration with Anton Gartner group at the University of Dundee and WTSI Cancer Genome Project group, and supported by the Consortium for Mutational Signatures (COMSIG).

\end{document}

